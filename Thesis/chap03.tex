\chapter{Analysis}

\section{Game Engine}

- unity because of familiarity

- mention the features we will use

\section{Procedural Level Generation}

- make each level unique

- one random seed dictates whole run

- in background thread

- debug visualizations (images)

\section{Random Number Generation}

- unity random X

- system random X

- custom random \checkmark

- why LCG

\section{Path Generation}

- how to make paths with the required qualities?

\subsection{Initial Paths}

- generate fixed number of paths with given lengths

- these are just plans to ensure paths of required length exist

- first pick *start points* from along the edges of the level (all paths end in the center of the level)

- choose randomly from positions with the correct parity (even / odd path length)

- spread them out by removing all positions near already chosen one

- then generate paths

- first trace it randomly, with bias away from the start and from path tiles

- simulated annealing

- there are just plans to make sure the paths exist and the terrain in a way that ensures that the paths are not blocked

\subsection{Final Paths}

- after terrain generation, the paths are traced for real

- guaranteed shortest length

- DFS, find all branches, but continue from bottom of the stack when a paths is found

\subsection{Path Visualization}

- line renderer \checkmark

- why does it look this way

\section{Terrain Generation}

- fractal noise X

- WFC \checkmark

\xxx{this is gonna be multiple subsections}

\xxx{illustrate with images (from the videos I made)}

- *slots* offset compared to *tiles*

- less distinct variants

- more control over transitions

- prepare *modules*

- generate grid of *slots*, mark them as *uncollapsed*

- each *slot* can become one of many *modules*

- the *modules* that can be placed in a given slot are its *domain*

- at the start each *slot* has all *modules* in its *domain*

- from the *domain*, compute all possible *boundary conditions*

- for example - there is a *module* in the *domain* with a cliff on its east boundary, so mark cliff to the east as possible

- mark all *slots* as *dirty*

- then repeat:

- propagate constraints

- for each *dirty slot*, until there are none:

- mark as *not dirty*

- find out which *modules* from its *domain* can be placed here and remove the rest from its *domain*

- decide only by neighbors' (orthogonal and diagonal) *boundary conditions*

- update *boundary conditions*

- if *boundary conditions* changed, mark all *uncollapsed* neighbors as *dirty*

- *collapse* a slot

- save the current *state* of all *slots* on a stack

- pick a *slot*

- pick one *module* from its *domain* at random

- weighted - provides control

- remove each other *module* from its *domain*

- update *boundary conditions*

- mark *uncollapsed* neighbors as *dirty*

- if a *slot* ends up with 0 *modules* in its *domain*, backtrack

- pop a prevoiusly saved *state* from the stack and revert to it

- remove the previously chosen *module* from the *domain* of the previously *collapsed slot*


- Which slot to collapse?

- fail fast approach

- prioritize slot with least options

- changed to slot with least entropy, because that's more accurate

- slots near most constraining modules were prioritized, making them more common and leading to repetitive terrain features

- better to just collapse a random slot

- in the end still weighted by entropy

- at first I tried to prefer slots with more entropy

- define overall structure first by sparsely covering the world, then fill in details

- often led to deep dead-ends with a lot of backtracking

- in the end, tiles with less entropy are preferred

- Limited backtracking depth

- usually when more backtracking is required, the search would take too long and it's faster to restart the algorithm

\section{Resources and Obstacles}

- after terrain generation, place blockers on tiles

- materials for the player to mine

- just rocks for variety - the player cant build on these

- set up as a few stages

- each stage has:

- one type of blocker (e.g. ore, small rocks, big rocks)

- *min* and *max* amounts

- *base chance* to place

- whether they can be placed on slanted tiles

- which Terrain Types they can be on (currently there is only one)

- *forces* - effect on chance based on already placed blockers

- for example: negative force with magnitude *m* from stage *s* means the chance to place a blocker on a given tile is decreased by *m/d*  for each blocker placed in stage *s*, where *d* is its distance from the considered tile

- for each stage:

- repeat until at least min blockers have been placed (in this stage)

- for each tile without a path or blocker (in random order):

- if random number between 0 and 1 < modified chance:

- place the blocker of the given type

- if there are max blockers (placed in this stage), end the stage

- scattering

- unity physics engine X

- parallel

For the blockers, I didn't want repetitive obstacle models, so they are generated proceduraly by scattering many simpler models (decorations) on each tile

- first compute weights based on various factors (images!!!)

- distance to path

- height

- distance to other blockers

- customizable thanks to modular approach

- then scatter decorations in stages, each stage again having one type of decoration and many parameters

- for each tile in random order repeat x (specified for this stage) times:

- pick a random position within it

- calcualte the weight at this position (based on settings)

- check that it is greater than some threshold (based on settings)

- calculate the minimum distance to other decorations (from weight, based on settings)

- check that the position is far enough from other decorations

- calculate the decoration size (from weight, based on settings)

- place the decoration on this position, with the given size

\section{Terrain Types}

- what information is tied to the type

- why txt (inspector was not as legible)

\section{World Builder}

- builds the world from the generated data, it needs to be done in the main thread

\section{Attacker Wave Generation}

- creates a randomized plan of waves

- two types of waves

- combine different attackers in sequence

- combine different attackers in parallel (only possible with multiple paths, rarer)

- each wave gets some throughput budget and buffer

- each attacker has a given cost

- when planning a wave, select attackers and spacing, such that the througput budget is exceeded

- for each attacker subtract the throughput overshoot from buffer

- fit such that as much of buffer gets used without going over

\section{Simulation}

- use fixed updates for game logic

- why?

- 20Hz = fixed time step 0.05s

- options to speed up or possibly pause - changing fixed update rate - not yet implemented

\section{Visuals and Interpolation}

- interpolate positions and visuals on Update

- many visuals are game-speed agnostic     - TODO: use unscaledDeltaTime

- I thought about some custom mini-framework for this, but many of the simulated variables the visuals are based on should be handled on case-by-case basis

\section{Attacker Targeting}

- Towers use it to acquire targets

- handles which Attackers are in range and which one is chosen as the current target

- can require line of sight to the enemy

- different targeting types

- rotation

- heights

- possibly ensure a trajectory

- preferred target (configurable)

\section{Range Visualization}

- IMAGES!!

- Draw the range on the terrain mesh

- Draw on which parts of paths will Attackers be targeted

- green - all sizes

- yellow - only large

- Terrain shader uses compressed texture format instead of raw texture

- Options:

- quadrant compression format, 2bytes per node

- less CPU time, because the data is already in this format

- up to 48KiB per frame

- more GPU time

- 256x256 texture, 1byte per pixel

- more CPU time

- 64KiB per frame

- fast on GPU

- only 1 channel - cannot interpolate

- mipmaps -> one additional state

- less CPU time

- 33\% more data

- more pixels per byte

- possible future optimization

- less data

- more difficult indexing and stuff both on CPU and GPU

- interpolation could work with more than one channel and without mipmaps

\section{Game Commnds}

- we want various components to modify how other components function

- examples

- also react to events as a bonus

\section{Blueprints}

- why are they implemented this way

\subsection{Attacker Stats}

- blueprints for attackers

\subsection{Dynamic Descriptions}

- explain what things do and their stats

- attackers and blueprints

- dynamically reflect the changes made by other components
