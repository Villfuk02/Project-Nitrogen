\chapter{Introduction}

\todo{Games exist. They are good. I like games. I will make a game.}

\section{Game Genre}

There are a plethora of games. Each is unique in its own way, but there are many similarities among them. One of the ways to categorize games is by their genre. A genre can encompass many characteristics of a game, most often its mechanics, but also its theme, art style or the medium it is played on. Genres have no exact definitions or strict boundaries and similarly, any individual game is usually a mix of different genres.

\subsection{Strategy}

One major genre is strategy games. Strategy games focus on tactics and long term planning. They require a lot of thinking. There are various kinds of strategy games, but most often, players compete against each other to reach some goal. These players can be real humans or artificial intelligence agents. Strategy games often utilize hidden information or rely on the unpredictability of other players' actions to create an environment where there is no single best way to reach the goal. This means that players have to have a good understanding of the game and be able to adapt to the situation at hand.

There are many qualities players enjoy in strategy games. Of course, it feels great to outsmart your opponent or conquer a challenge. But strategy games also provide a sense of progression and accomplishment because of their depth. They are often very replayable because of the different situations that can arise from the game's mechanics.

One way to categorize a strategy game is whether is it real-time or turn-based. In turn-based strategy, players take turns to make their moves. This allows for a slower pace and more time to think about the best move. In real-time strategy games, however, the environment evolves continuously and players have to react and make decisions quickly.

\subsection{Tower Defense}

Tower defense is a subgenre of real-time strategy where the player has to defend against waves of enemies. The player has to build towers, which attack the enemies as they approach, in order to defend their base. The attackers are very predictable and usually follow a set path. The player has to build their towers in a way that maximizes their effectiveness against the attackers.
\?{do I want images and examples}

The game usually consist of multiple levels, each presenting a different challenge. The player has to adapt to the different attackers and the different terrain. Sometimes the levels make up a campaign, where the player has to progress through harder and harder levels to reach the end. Other times, the levels are standalone and the player can choose which level to play, where the goal can be to survive as long as possible. This can take a few hours and may become very repetitive.

The attackers can have unique abilities or be resistant to certain types of towers. The composition of each wave can be predetermined or randomized to a varying degree. In this way, the game can force the player to adapt and use different strategies. To make their decisions interesting, the player has various towers and upgrades at their disposal, each with different abilities, often complimenting each other.

Players can get resources to build their defense passively, but some tower defense games also include an economy system. Here, the player has to build economic buildings to generate the resources. This adds another layer of strategy to the game, as the player has to balance their economy with their defense.

\subsection{Rogue-like}

Rogue-like is a subgenre of role-playing games. In role-playing games, the player takes on the role of a character and goes on an adventure. The character can grow stronger by acquiring new abilities, items or experience. The player has to make decisions about how to upgrade their characters in order to overcome the challenges they might face. This high level of customization and the sense of progression makes role-playing games very engaging.

Rogue-like is named after the game Rogue, released in 1980\?{citation!}. The game was known for its high difficulty and the fact that the player had to start from the beginning if they died. The game is usually procedurally generated, meaning that the levels are created by an algorithm, rather than being designed by a human. This is important, because the player can't memorize the levels and has to rely on their skill and knowledge of the game's mechanics to progress. Since all in-game progress is lost when the player dies, the player has to learn from their mistakes and improve their skill in order to advance further. These are defining characteristics of rogue-likes which separate them from other role-playing games.

In most rogue-like games, you explore a dungeon, where you have to fight enemies and avoid traps. You are then rewarded with items and other resources to improve your character. The game is usually turn-based, making it more strategic.

There are many games which break the traditional rogue-like formula, but still share many of its characteristics. They might have a different theme, where you don't even fight monsters but instead compete in another task. They might include a progression system, where you unlock new items or upgrades for your future runs. Real-time gameplay is also common, where more skills are tested than just the player's ability to think strategically. These games are often called rogue-lites, however, we will not make such a distinction since all genres can be bent and blended in many ways. \?{do I include more examples here}

\subsection{Combining the two}

I enjoy playing both rogue-likes and tower defense games a lot\?{should this be here}. There aren't many games which combine these two genres. It is possible that this combination doesn't work well, however, it is worth exploring, because these two genres seem to compliment each other. This is why this game will be a blend of these two genres.

The main gameplay loop will be a tower defense game. The player will play through many short levels. Tower defense can get a bit stale, if you find a strategy that just works and you can use it every time. This is how some rogue-like elements can help. Each level will be procedurally generated featuring different terrain and attackers. Additionally, the player will start with a small arsenal of towers and other blueprints, and acquire new ones from a randomized selection as they progress. This means that the player will likely use a different strategy in each run.

\section{Original vision}

- player hops from planet to planet with their spaceship, stopping to refuel on each one. Here they have to defend until they get enough fuel to move on

- build economic buildings, build towers, defend against waves - balance economy, defense and offense

- each planet will have different terrain and attackers

- slowly, they will acquire new blueprints for defensive towers, economic buildings and other upgrades. These will be randomized, so you can't rely on the same strategy every time

- along the way, there will be events, shops, harder battles and other anomalies that the player can choose to interact with

- there will be some sort of story motivating why the player is going on this perilous journey

\section{Current scope}

- a prototype

- playable battles - we can start play testing the core gameplay

- all necessary battle systems to add more content later

- progressing through levels and collecting blueprints to test which blueprints work well (provide new strategies, are balanced)

\subsection{Goals}
