\chapter{Introduction}

\todo{Games exist.
    They are good.
    I like games.
    I will make a game.}

\section{Game Genre}

There are a plethora of games.
Each is unique in its own way, but there are many similarities among them.
One of the ways to categorize games is by their genre.
A genre can encompass many characteristics of a game, most often its mechanics, but also its theme, art style or the medium it is played on.
Genres have no exact definitions or strict boundaries and similarly, any individual game is usually a mix of different genres.

\subsection{Strategy}

One major genre is strategy games.
Strategy games focus on tactics and long-term planning.
They require a lot of thinking.
There are various kinds of strategy games, but most often, players compete against each other to reach some goal.
These players can be real humans or artificial intelligence agents.
Strategy games often utilize hidden information or rely on the unpredictability of other players' actions to create an environment where there is no single best way to reach the goal.
This means that players have to have a good understanding of the game and be able to adapt to the situation at hand.

There are many qualities players enjoy in strategy games.
Of course, it feels great to outsmart your opponent or conquer a challenge.
But strategy games also provide a sense of progression and accomplishment because of their depth.
They are often very replayable because of the different situations that can arise from the game's mechanics.

One way to categorize a strategy game is whether is it real-time or turn-based.
In a turn-based strategy, players take turns to make their moves.
This allows for a slower pace and more time to think about the best move.
In real-time strategy games, however, the environment evolves continuously and players have to react and make decisions quickly.

\subsection{Tower Defense}

Tower defense is a subgenre of real-time strategy where the player has to defend against waves of enemies.
The player has to build towers, which attack the enemies as they approach, to defend their base.
The attackers are very predictable and usually follow a set path.
The player has to build their towers in a way that maximizes their effectiveness against the attackers.
\?{do I want images and examples}

The game usually consists of multiple levels, each presenting a different challenge.
The player has to adapt to the different attackers and the different terrain.
Sometimes the levels make up a campaign, where the player has to progress through harder and harder levels to reach the end.
Other times, the levels are standalone and the player can choose which level to play, where the goal can be to survive as long as possible.
This can take a few hours and may become very repetitive.

The attackers can have unique abilities or be resistant to certain types of towers.
The composition of each wave can be predetermined or randomized to a varying degree.
In this way, the game can force the player to adapt and use different strategies.
To make their decisions interesting, the player has various towers and upgrades at their disposal, each with different abilities, often complimenting each other.

Players can get resources to build their defense passively, but some tower defense games also include an economy system.
Here, the player has to build economic buildings to generate the resources.
This adds another layer of strategy to the game, as the player has to balance their economy with their defense.

\subsection{Rogue-like}

Rogue-like is a subgenre of role-playing games.
In role-playing games, the player takes on the role of a character and goes on an adventure.
The character can grow stronger by acquiring new abilities, items or experiences.
The player has to make decisions about how to upgrade their characters to overcome the challenges they might face.
This high level of customization and the sense of progression makes role-playing games very engaging.

Rogue-like is named after the game Rogue, released in 1980\?{citation!}.
The game was known for its high difficulty and the fact that the player had to start from the beginning if they died.
The game is usually procedurally generated, meaning that the levels are created by an algorithm, rather than being designed by a human.
This is important because the player can't memorize the levels and has to rely on their skill and knowledge of the game's mechanics to progress.
Since all in-game progress is lost when the player dies, the player has to learn from their mistakes and improve their skill to advance further.
These are defining characteristics of rogue-likes that separate them from other role-playing games.

In most rogue-like games, you explore a dungeon, where you have to fight enemies and avoid traps.
You are then rewarded with items and other resources to improve your character.
The game is usually turn-based, making it more strategic.

Many games break the traditional rogue-like formula but still share many of its characteristics.
They might have a different theme, where you don't even fight monsters but instead compete in another task.
They might include a progression system, where you unlock new items or upgrades for your future runs.
Real-time gameplay is also common, where more skills are tested than just the player's ability to think strategically.
These games are often called rogue-lites, however, we will not make such a distinction since all genres can be bent and blended in many ways.
\?{do I include more examples here}

\subsection{Combining the Two}

I enjoy playing both rogue-likes and tower defense games a lot\?{should this be here}.
There aren't many games that combine these two genres.
It is possible this combination doesn't work well, however, it is worth exploring, because these two genres seem to complement each other.
This is why this game will be a blend of these two genres.

The main gameplay loop will be a tower defense game.
The player will play through many short levels.
Tower defense can get a bit stale if you find a strategy that can be used every time.
This is how some rogue-like elements can help.
Each level will be procedurally generated featuring different terrain and attackers.
Additionally, the player will start with a small arsenal of blueprints --- defensive towers, abilities and economic buildings.
They will acquire new ones from a randomized selection as they progress.
This means that the player will likely use a different strategy in each run.

\section{Original Vision} \label{sec:original-vision}

The game will be for the PC.
This is because it is going to be pretty complex and PCs usually have screens large enough to display all the information the player needs.
The mouse and keyboard allow for precise control and a lot of buttons to be used.
PC players are also more likely to enjoy challenging strategy games, which is what this game aims to be.

It will be a single-player game.
As stated, the small-scale gameplay will be a tower defense, but on a larger scale, the game will be rogue-like.
This means that it will consist of individual runs, where the player will start from scratch every time.
Their goal is to get as far as possible, trying to reach the final level and beat the game.

The game should have some story to motivate the player on their journey.
The details of this story are not yet decided, but the game will likely be set in space.
The player will travel forward through an unexplored galaxy trying to reach a destination specified by the story.
The levels will take place on different planets, each with its own theme and unique terrain.
This sci-fi theme allows for a lot of freedom in the designs of the levels, buildings and attackers.
It also works well with the mechanics of the individual levels or, as we will call them "battles".

The goal of a single battle is to gather enough fuel to continue to the next planet.
The faster the player gathers the fuel, the sooner they win the battle.
Fuel is gathered passively, but additional buildings can be built to speed up the process.
In the meantime, the player has to defend against waves of attackers by building towers and using abilities.
All of this costs materials and energy --- resources, which are generated by economic buildings.
Thus, the player will have to balance their use of resources among defense, economy and gathering fuel.

On their way, the player will encounter various events, shops and other anomalies.
These will provide the player with opportunities to gain resources, blueprints or other bonuses.
They will also provide the player with choices, which can have different outcomes.
The player will have to decide which choice is the best for their current situation.
Some information about the upcoming battles will be given to the player, for example, the attacker types and general difficulty of the battle.
Harder battles will provide better rewards.
The path the player takes will be non-linear allowing them to decide which battles to fight and what to interact with.
This way they can choose to take a bigger risk for a bigger reward or play it safe.

\todo{I probably forgot to mention something}

\section{Current Scope and Goals}

The scale of the full game is quite large.
Too much for the scope of a bachelor's thesis.
Furthermore, we will not consider the story, art or sound design as that is not our area of expertise.
Instead, the goal will be to create a functional prototype, which can be used to playtest the core gameplay.
It will require some basic content --- several attackers and blueprints --- so the game can be properly tested.
The prototype will be prepared for future development so that more content can be added later.

Playtesting is very important because it can provide valuable feedback about the game's mechanics.
This prototype will allow for testing the game and adjusting the design, tweaking the mechanics or making things more clear to the player.
It is important to start with playtesting as
soon as possible to avoid wasting time on content that doesn't work.

The prototype will consist of all systems and mechanics necessary to play through a battle.
This includes attackers, towers, abilities and the economy.
It will also include procedural generation of the battles including the terrain, attacker paths and the makeup of the attacker waves.
The player will be able to progress through battles and collect blueprints.
But there will be no map view to choose their path because the progression will be linear.
There will also be no events or shops.

All the art and sound assets will be placeholders.
But care will be taken to make everything as clear as possible to the player.

Additionally, the prototype will include a very rudimentary tutorial to explain the game's mechanics to the player.
This is important because the game will be complex and the player needs to understand how the game works to be able to play it properly.

The main goal of this thesis is to design, and implement this prototype of the game, so it can be playtested.
It will also focus on the design of the game's mechanics and systems, and the decisions behind them.
The game will be implemented in the Unity game engine, using C\# as the programming language.
The approaches to implementing the game's mechanics and systems will be discussed in detail.
Notably, the algorithms that were used and why they were chosen.
