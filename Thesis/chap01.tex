\chapter{Introduction}

Video games are a popular form of entertainment.
There is a plethora of games to choose from, each offering a different experience.
Still, it is always possible to create something new that players might enjoy.
The author of this thesis enjoys both \emph{tower defense} games and \emph{roguelike} games and there are not many games that combine these two genres.
In this thesis, we will design and implement a video game, that uniquely blends them, and discuss the decisions behind it.
So, what do we mean by a \emph{roguelike tower~defense} game?

\section{Tower Defense}

\textbf{Tower defense} is a subgenre of \emph{real-time strategy}.
This means the game focuses on long-term planning, but also quick thinking.
In \emph{tower defense} games, the player has to defend against waves of attackers by building defensive towers.
As an example we'll look at \emph{Plants vs.\ Zombies}\cite{PvZWiki}, released in~2009.

\begin{center}
    \captionsetup{type=figure}
    \includegraphics[width=0.8\textwidth]{img/Plants-vs-Zombies-Fight.png}
    \caption{A level in \emph{Plants vs.\ Zombies}. On the left there are some \emph{Sunflowers} and two \emph{Repeaters}, one of them shooting at a zombie to the right.}
    \label{fig:pvz-fight}
\end{center}

In \emph{Plants vs.\ Zombies}, the player defends their house from zombies.
The goal of each level is to survive all the incoming waves by placing plants that fire projectiles or otherwise impede the zombies.
The zombies come from the right side of the screen and advance left, each moving only along its lane.
If any zombie reaches the far left edge of the screen, the player loses the level.
Placing plants costs~\emph{sun}, and the player can plant \emph{Sunflowers} to produce more.
In our game, the player will also build towers, to defend from waves of attackers, and economic buildings that produce materials.

Though, it will differ a lot from \emph{Plants vs.\ Zombies} in the overall structure of the game.
The main game mode of \emph{Plants vs.\ Zombies} is a campaign consisting of~50 individual levels.
If the player loses a level, they can try again and~again until they succeed in beating it.
After most levels, the player unlocks a new plant, which they can use in upcoming levels, slowly building up their arsenal.
In our game, however, once the player loses, they lose all their progress and must start from the very beginning.
This and some other mechanics are taken directly from the \emph{roguelike} genre.

\section{Roguelike}

\textbf{Roguelike} is a subgenre of \emph{role-playing games}.
In \emph{role-playing games}, the player takes on the role of a character and goes on an adventure.
The character can grow stronger by acquiring new abilities, items, or~experiences.
The player has to make decisions about how to upgrade their characters to overcome the challenges they might face.

\emph{Roguelike} is named after the game \emph{Rogue}\cite{RogueWiki}, released in~1980.
In this single-player turn-based game, the player explores a grid-based dungeon and fights monsters that inhabit~it.
Along the way, they collect various weapons, armor and other magical items that improve their abilities.
It features \emph{permadeath}, which means that when the character dies, the player loses all progress and must start from the very beginning.
The dungeon is randomized---it is different in every run, so the player can't just memorize the layout.

A more recent game that's a good example of this genre is \emph{Slay the Spire}\cite{StSWiki}, published in~2019.
In \emph{Slay the Spire}, the player ascends a spire and fights various enemies.
The fights are also turn-based, and when the player's character dies, they have to start from scratch.
However, it is not a traditional \emph{roguelike}.
The game is not played on a grid, instead the spire the player navigates is~a graph of separate rooms, where they move from bottom up.
There are different kinds of rooms, but the most important are enemy encounters.
In these the player fights using a deck of cards.

\begin{center}
    \captionsetup{type=figure}
    \includegraphics[width=0.8\textwidth]{img/Slay-the-Spire-Map.png}
    \caption{The map screen in \emph{Slay the Spire}. Each icon is a room. The player has been to the circled ones and can choose where to go next based on the dashed lines connecting the rooms.}
    \label{fig:sts-map}
\end{center}

At the start of each turn, the player draws five cards from the deck.
Each card has an \emph{energy}~cost, and it has some effect that's written on it.
Most cards deal damage to the enemies or provide \emph{block} to defend from enemy attacks, but some have more unique effects.
The player can only spend three \emph{energy} per turn, so they can only play a limited amount of the cards they drew.
It is important to play the right cards in order to kill the enemy without taking a lot of damage.

\begin{center}
    \captionsetup{type=figure}
    \includegraphics[width=0.8\textwidth]{img/Slay-the-Spire-Fight.png}
    \caption{A fight in \emph{Slay the Spire}. The player character on the left is facing a \emph{Jaw worm} on the right of the screen. On the bottom, there are cards that the player can play to fight the enemy.}
    \label{fig:sts-fight}
\end{center}

Even though the player never knows exactly what cards they'll draw, they can shape the deck they draw from.
The player starts each run with a predefined deck of starter cards, and as they progress, they add new cards into their deck.
For example, after every fight, they get presented with three randomly selected cards, and they can choose one of them.
The player can also get new cards from events or shops and sometimes remove the cards they don't want.

Some cards are rarer than others, and they are often more powerful.
However, being lucky ang getting the most powerful cards is not what the game's about.
The player must learn which cards work together well and which don't, and understand the weaknesses of their deck and how to fix them.

Many games take the \emph{roguelike} mechanic of \emph{permadeath} and randomized procedural generation, but fill in different game mechanics.
\emph{Slay the Spire} has the player build their own deck of cards to play with, but they still play as a character that fights enemies.
Some, however, deviate much more.
In our game, the battles will be in the style of \emph{tower defense}, and the player will collect blueprints for defensive towers and other buildings instead of weapons and armor.

Games that deviate more from the \emph{roguelike} formula are sometimes called \emph{roguelite} games.
However, there is no agreement on when a game stops being \emph{roguelike} and starts being \emph{roguelite}.
We will not make this distinction, since game genres have no precise boundaries and can be freely blended with others.

\section{Original Vision} \label{sec:original-vision}

The game will be single-player.
As stated, the moment to moment gameplay will be a \emph{tower defense}, but on a larger scale, the game will be \emph{roguelike}.
This means that it will consist of individual procedurally generated runs, where the player will start from scratch every time.
During each run, the player will defend against attackers in many battles and improve their arsenal to grow stronger.
Their goal is to get as far as possible, trying to reach the final level and beat the game.

The goal of each battle is to gather enough \emph{fuel} to continue.
The faster the player gathers the \emph{fuel}, the sooner they win the battle.
The \emph{fuel} is generated passively, but additional buildings can be built to speed up the process.
In the meantime, the player has to defend against waves of attackers by building towers and using abilities.
All of this costs \emph{materials} and \emph{energy}---resources, which are generated by economic buildings.

On their way, the player will choose from randomly selected \emph{blueprints} to add to their collection.
These \emph{blueprints} will be for new abilities, towers or other buildings.
The player will have to choose blueprints which work together well in order to use their full potential.

The player will also encounter various shops and events.
These can present additional choices and provide the player with opportunities to gain various rewards or punishments.
The path the player takes will not be linear, allowing them to decide which battles to fight and what to interact with.
This is important, because some battles will be harder, but provide better rewards.

The game will be for personal computers.
Unlike mobile phones, PCs usually have a screen large enough to let us clearly convey all the information the player needs.
It won't be for game consoles either because we think a mouse will be the best way to control the game.
The mouse allows the player to select a precise position in the world quickly.
The player will also control certain aspects of the game using the keyboard.

\section{Current Scope and Goals}

The scale of the full game is quite large, too much for the scope of a bachelor's thesis.
Our goal will be to create a functional prototype, which can be used to playtest the core gameplay.
It will require some base content, so the game can be properly tested.
The prototype will be prepared for future development so that more content can be added later.

The prototype will consist of all systems and mechanics necessary to play through a battle.
This includes attackers, towers, abilities, and the economy.
It will also include procedural generation of the battles including the terrain, attacker paths, and the makeup of the attacker waves.
The player will be able to progress through battles and collect blueprints.
However, there will be no map view to choose their path because for now, the progression will be linear.
There will also be no events or shops, only battles.
All the art and sound assets will be placeholders, but care will be taken to make everything as clear as possible to the player.
Additionally, the prototype will include a very rudimentary tutorial to explain the game's mechanics to the player \xxx{(this might be false)}.
