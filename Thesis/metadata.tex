%%% Please fill in basic information on your thesis, which will be automatically
%%% inserted at the right places.

% Type of your thesis:
%	"bc" for Bachelor's
%	"mgr" for Master's
%	"phd" for PhD
%	"rig" for rigorosum
\def\ThesisType{bc}

% Language of your study programme:
%	"cs" for Czech
%	"en" for English
\def\StudyLanguage{cs}

% Thesis title in English (exactly as in the official assignment)
% (Note: \xxx is a "ToDo label" which makes the unfilled visible. Remove it.)
\def\ThesisTitle{Tower Defense Game with Procedurally Generated Content and Rogue-like Elements}

% Author of the thesis (you)
\def\ThesisAuthor{Vilém Gutvald}

% Year when the thesis is submitted
\def\YearSubmitted{2024}

% Name of the department or institute, where the work was officially assigned
% (according to the Organizational Structure of MFF UK in English,
% see https://www.mff.cuni.cz/en/faculty/organizational-structure,
% or a full name of a department outside MFF)
\def\Department{Department of~Distributed and~Dependable Systems}

% Is it a department (katedra), or an institute (ústav)?
\def\DeptType{Department}

% Thesis supervisor: name, surname and titles
\def\Supervisor{Mgr.~Pavel~Ježek,~Ph.D.}

% Supervisor's department (again according to Organizational structure of MFF)
\def\SupervisorsDepartment{Department of~Distributed and~Dependable Systems}

% Study programme (does not apply to rigorosum theses)
\def\StudyProgramme{Computer Science}

% An optional dedication: you can thank whomever you wish (your supervisor,
% consultant, who provided you with tea and pizza, etc.)
\def\Dedication{%
    Chtěl bych poděkovat svému vedoucímu práce, Mgr.~Pavlu~Ježkovi,~Ph.D., za jeho vedení a cenné rady.
    Dále děkuji svým přátelům, kteří testovali hru a poskytli spoustu užitečné zpětné vazby.
    Jejich nadšení mi dalo naději, že hra, kterou jsem vytvořil v rámci této práce, má potenciál být velmi dobrá.
    Také jsem nesmírně vděčný svým rodičům za jejich podporu a za to, že mi poskytli zázemí, nejen v průběhu psaní této práce.
}

% Abstract (recommended length around 80-200 words; this is not a copy of your thesis assignment!)
\def\Abstract{%
    In this thesis, we designed and implemented in the Unity game engine a~demo version of a rogue-like tower defense game.
    We employed various procedural generation techniques, including wave function collapse and simulated annealing, to generate level terrain and attacker paths.
    We also developed an algorithm to~procedurally generate attacker wave composition.
    We implemented the primary gameplay systems, including resource management, tower and production building placement, special attacker abilities, and a blueprint collection system.
    We also created a simple tutorial to guide new players.
    Finally, we conducted a playtest to gather user feedback, verifying our design choices and identifying key areas for improvement, such as the user interface and the resource economy.
}

% 3 to 5 keywords (recommended) separated by \sep
% Keywords are useful for indexing and searching for the theses by topic.
\def\ThesisKeywords{%
    procedural generation\sep game development\sep game design\sep tower defense\sep rogue-like
}

% If any of your metadata strings contains TeX macros, you need to provide
% a plain-text version for use in XMP metadata embedded in the output PDF file.
% If you are not sure, check the generated thesis.xmpdata file.
\def\ThesisAuthorXMP{Vilem Gutvald}
\def\ThesisTitleXMP{\ThesisTitle}
\def\ThesisKeywordsXMP{\ThesisKeywords}
\def\AbstractXMP{\Abstract}

% If your abstracts are long and do not fit in the infopage, you can make the
% fonts a bit smaller by this setting. (Also, you should try to compress your abstract more.)
\def\InfoPageFont{}
%\def\InfoPageFont{\small}  % uncomment to decrease font size

% If you are studing in a Czech programme, you also need to provide metadata in Czech:
% (in English programmes, this is not used anywhere)

\def\ThesisTitleCS{Tower defense hra s~procedurálně generovaným obsahem a rogue-like prvky}
\def\DepartmentCS{Katedra distribuovaných a~spolehlivých systémů}
\def\DeptTypeCS{Katedra}
\def\SupervisorsDepartmentCS{Katedra distribuovaných a~spolehlivých systémů}
\def\StudyProgrammeCS{Informatika}

\def\ThesisKeywordsCS{%
    procedurální generování\sep vývoj her\sep game design\sep tower defense\sep rogue-like
}

\def\AbstractCS{%
    V této práci jsme navrhli a implementovali v herním enginu Unity demo verzi rogue-like tower defense hry.
    Využili jsme různé techniky procedurálního generování, včetně wave function collapse a simulovaného žíhání, ke generování terénu a cest pro útočníky.
    Také jsme vyvinuli algoritmus pro procedurální generování složení vln útočníků.
    Implementovali jsme hlavní herní systémy, včetně správy surovin, umisťování věží a výrobních budov, speciálních schopností útočníků a systému sbírání nákresů nových budov.
    Dále jsme vytvořili jednoduchý tutorial, který naučí hráče, jak hru hrát.
    Nakonec jsme provedli playtest k získání zpětné vazby, který nám umožnil ověřit návrhová rozhodnutí a identifikovat důležité oblasti pro zlepšení, například uživatelské rozhraní a systém správy surovin.
}
